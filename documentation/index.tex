%%%%%%%%%%%%%%%%%%%%%%%%%%%%%%%%%%%%%%%%%%%%%%%%%%%%%%%%%%%%%%%%%%%%%%%%%%%%%%%%%%%%%%%%%%%%%%%%%%%
%
% Primordial Machine's Ranges Library
% Copyright (C) 2017-2019 Michael Heilmann
%
% This software is provided 'as-is', without any express or implied warranty.
% In no event will the authors be held liable for any damages arising from the
% use of this software.
%
% Permission is granted to anyone to use this software for any purpose,
% including commercial applications, and to alter it and redistribute it
% freely, subject to the following restrictions:
%
% 1. The origin of this software must not be misrepresented;
%    you must not claim that you wrote the original software.
%    If you use this software in a product, an acknowledgment
%    in the product documentation would be appreciated but is not required.
%
% 2. Altered source versions must be plainly marked as such,
%    and must not be misrepresented as being the original software.
%
% 3. This notice may not be removed or altered from any source distribution.
%
%%%%%%%%%%%%%%%%%%%%%%%%%%%%%%%%%%%%%%%%%%%%%%%%%%%%%%%%%%%%%%%%%%%%%%%%%%%%%%%%%%%%%%%%%%%%%%%%%%%

\documentclass[oneside]{book}

\input{common}

\SetOrganization{Primordial Machine's}
\SetLibraryName{Ranges Library}
\SetLibraryVersion{v1.3}
\SetLibraryRepository{https://github.com/primordialmachine/ranges}
\SetAuthor{Michael Heilmann}
\SetEmail{michaelheilmann@primordialmachine.com}

\begin{document}
\maketitle
\tableofcontents
\chapter{Synopsis}
\textit{\GetOrganization{} \GetLibraryName} is a C++ 17 library providing ranges.
The library is made available publicly on
\href{\GetLibraryRepository}{Github}
under the
\href{\GetLibraryRepository/blob/master/LICENSE}{MIT License}.

\chapter{Limitations and Restrictions}
The library officially only supports Visual Studio 2017 and Windows 10.

\chapter{Introductory example}
Examples are located in the \href{\GetLibraryRepository/blob/master/examples}{examples} directory.

\input{building_visual_studio_2017}

\chapter{Example}
The following example creates a \verb+iterator_range+ from the iterators returned by the
\verb+cbegin+ and \verb+cend+ functions of a \verb+std::vector<int>+.
\begin{verbatim}
#include "primordialmachine/ranges/include.hpp"
#include <vector>
#include <stdlib.h>

int main(int argc, char **argv) {
  using namespace primordialmachine;
  using namespace std;
  using collection_type = vector<int>;
  static const auto collection = collection_type{ 3, 5, 7 };
  auto range = iterator_range<collection_type::const_iterator>(primes.cbegin(), primes.cend());
  for (auto iterator = range.begin(); iterator != range.end(); ++iterator) {
    std::cout << (*iterator) << std::endl;
  }
  for(const auto& element : range) {
    std::cout << element << std::endl;
  }
  return EXIT_SUCCESS;
}
\end{verbatim}


\chapter{Library Interface Documentation}

\section{\texttt{namespace primordialmachine}}
The namespace this library is adding its declarations/definitions to.
The added namespace elements are documented below.

\section{\texttt{template<typename ITERATOR> class iterator\_range}}
\texttt{iterator\_range} is a generic range over two iterators.
It has a single template parameter \texttt{typename ITERATOR} denoting the type of the     iterator(s)
which must be an \href{https://en.cppreference.com/w/cpp/named_req/InputIterator}{InputIterator} type.
\texttt{iterator\_range} is \textit{Assignable} and \textit{Copyable}. The class provides         the
following members.

\subsection{\texttt{using iterator\_type}}
\texttt{iterator\_range} provides the member type definition \texttt{iterator\_type} which is an alias
for \texttt{ITERATOR}.

\subsection{\texttt{iterator\_range()}}
default
constructs this \texttt{iterator\_range} object with
the value \texttt{iterator\_type()} as the iterator value to the start of the range and
the value \texttt{iterator\_type()} as the iterator value to the end of the range.

\subsection{\texttt{iterator\_range(iterator\_type begin, iterator\_type end)}}
constructs this \texttt{iterator\_range} object with
`begin` as the iterator to the start of the range and
`end` as iterator to the the end of the range.
Consequently, \texttt{begin() == begin}, \texttt{end() == end} all hold true.

\subsection{\texttt{iterator\_type begin() const}}
Get the iterator to the beginning of the range.

\subsection{\texttt{iterator\_type end() const}}
Get the iterator to the end of the range.

\subsection{\texttt{bool empty() const}}
Get if the range is empty.
Return \verb+true+ if the range is empty and \verb+false+ otherwise.

\end{document}
