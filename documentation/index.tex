\documentclass[oneside]{article}
\usepackage[utf8]{inputenc}
\usepackage{xcolor}
%
\usepackage{lmodern}
\usepackage{microtype}
%
\usepackage{graphicx}
\usepackage[colorlinks]{hyperref}
\usepackage[margin=0.75in]{geometry}
\usepackage{amsmath,amssymb,amsfonts}

\input{common}

\SetOrganization{Primordial Machine's}
\SetLibraryName{Ranges Library}
\SetLibraryVersion{v1.2}
\SetLibraryRepository{https://github.com/primordialmachine/ranges}
\SetAuthor{Michael Heilmann}
\SetEmail{michaelheilmann@primordialmachine.com}

\begin{document}
\maketitle
\tableofcontents
\section{Synopsis}
\textit{\GetOrganization{} \GetLibraryName} is a C++ 17 library providing ranges.
The library is made available publicly on
\href{\GetLibraryRepository}{Github}
under the
\href{\GetLibraryRepository/blob/master/LICENSE}{MIT License}.

\section{Limitations and Restrictions}
The library officially only supports Visual Studio 2017 and Windows 10.

\section{Introductory example}
Examples are located in the \href{\GetLibraryRepository/blob/master/examples}{examples} directory.

\section{Building under Visual Studio 2017}
\begin{enumerate}
\item Open the solution \texttt{range.sln} in Microsoft Visual Studio 2017.
\item Batch build everything.
\item The folder \texttt{packages} contains the distribution of the library i.e. include files and the
      static libraries for
  \begin{enumerate}
    \item the platforms \texttt{Win32} and \texttt{x64} and
    \item configurations \texttt{Release} and \texttt{Debug}.
  \end{enumerate}
\item Copy the contents of the \verb+packages+ folder into a directory. Let
      \verb+[library home]+ be a placeholder denoting the path by which that folder
      can be referenced from your project.
\item Add
  \begin{enumerate}
    \item the include path
\begin{verbatim}
[library home]/primordialmachine/ranges/
$(Platform.toLower())/$(Configuration.toLower())/includes
\end{verbatim}
	and
    \item the library path
\begin{verbatim}
[library home]/primordialmachine/ranges/
$(Platform.toLower())/$(Configuration.toLower())/libraries
\end{verbatim}
    to your project.
\end{enumerate}
\item Link your project with the library \verb+ranges.lib+.
\item Add the include directive \verb+#include "primordialmachine/ranges/include.hpp"+ where appropriate.
\item You can now use the functionality provided by the library.
\end{enumerate}

\section{Example}
The following example creates a \verb+iterator_range+ from the iterators returned by the
\verb+cbegin+ and \verb+cend+ functions of a \verb+std::vector<int>+.
\begin{verbatim}
#include "primordialmachine/ranges/include.hpp"
#include <vector>
#include <stdlib.h>

int main(int argc, char **argv) {
  using namespace primordialmachine;
  using namespace std;
  using collection_type = vector<int>;
  static const auto collection = collection_type{ 3, 5, 7 };
  auto range = iterator_range<collection_type::const_iterator>(primes.cbegin(), primes.cend());
  for (auto iterator = range.begin(); iterator != range.end(); ++iterator) {
    std::cout << (*iterator) << std::endl;
  }
  for(const auto& element : range) {
    std::cout << element << std::endl;
  }
  return EXIT_SUCCESS;
}
\end{verbatim}


\section{Library Interface Documentation}

\subsection{\texttt{namespace primordialmachine}}
The namespace this library is adding its declarations/definitions to.
The added namespace elements are documented below.

\subsection{\texttt{template<typename ITERATOR> class iterator\_range}}
\texttt{iterator\_range} is a generic range over two iterators.
It has a single template parameter \texttt{typename ITERATOR} denoting the type of the     iterator(s)
which must be an \href{https://en.cppreference.com/w/cpp/named_req/InputIterator}{InputIterator} type.
\texttt{iterator\_range} is \textit{Assignable} and \textit{Copyable}. The class provides         the
following members.

\subsubsection{\texttt{using iterator\_type}}
\texttt{iterator\_range} provides the member type definition \texttt{iterator\_type} which is an alias
for \texttt{ITERATOR}.

\subsubsection{\texttt{iterator\_range()}}
default
constructs this \texttt{iterator\_range} object with
the value \texttt{iterator\_type()} as the iterator value to the start of the range and
the value \texttt{iterator\_type()} as the iterator value to the end of the range.

\subsubsection{\texttt{iterator\_range(iterator\_type begin, iterator\_type end)}}
constructs this \texttt{iterator\_range} object with
`begin` as the iterator to the start of the range and
`end` as iterator to the the end of the range.
Consequently, \texttt{begin() == begin}, \texttt{end() == end} all hold true.

\subsubsection{\texttt{iterator\_type begin() const}}
Get the iterator to the beginning of the range.

\subsubsection{\texttt{iterator\_type end() const}}
Get the iterator to the end of the range.

\subsubsection{\texttt{bool empty() const}}
Get if the range is empty.
Return \verb+true+ if the range is empty and \verb+false+ otherwise.

\end{document}
